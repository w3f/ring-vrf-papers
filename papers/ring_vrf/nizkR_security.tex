% !TEX root = main.tex

\section{The Security of \nizkR}
\label{ap:nizkR}
\begin{theorem}
	If $\ZKCont_{\relone}$ is a zk continuation for $\relone$ and $\nizktwo$ is a NIZK for $\reltwo'$ for some appropriately chosen public parameters $\pp$, 
	then the $\nizkR$ construction described above is a NIZK for $\rel$.
\end{theorem}
\begin{proof} Putting together the results of Lemma~\ref{le:KS_for_nizkR}, Lemma~\ref{le:specialCompl_for_nizkR}, and
	Lemma~\ref{le:specialZK_for_nizkR}, we obtain the above statement.
\end{proof} 
\begin{lemma}[Knowledge-soundness for $\nizkR$] 
	\label{le:KS_for_nizkR}
	If $\ZKCont_{\relone}$ is a zk continuation for $\relone$ and $\nizktwo$ is a NIZK for $\reltwo'$ for some appropriately chosen public parameters $\pp$, 
	then the $\nizkR$ construction described above has knowledge-soundness for $\rel$. 
\end{lemma} 
%\vspace{-1cm}
\begin{proof}This is easy to infer by linking together the extractors guaranteed for $\ZKCont_{\relone}$ and $\nizktwo$ due to their respective 
	knowledge-soundness.
\end{proof}
%\vspace{-0.1cm}
\noindent Next, we define 
%\noindent {\color{red}\textbf{Perfect Completeness after reproving a $\ZKCont_{\relone}$ Proof}}
\emph{Special Perfect Completeness} for all $\lambda $, for every efficient adversary $\adv$, for every 
$(\barz, \barx; \baromega_2) \in \reltwo$ it holds
%{It is a completeness definition for $ \reltwo $ but there is nothing related to $ \reltwo $ below. We want to define a new completeness definition for $ 
	%\nizkR $. So, I suggest you to replace everything before given $ (|) $ with $ \nizkR.\Verify((\crsR,(\bary, \barz), (\pi_1, \pi_2, X)))\rightarrow 1 $ because %everything you have before $ | $are already satisfied from the $ \NIZK $ completeness and $ \ZKCont_{\relone} $ special completeness, so they are already %satisfied, they are not new statements. }: 
\eprint{\begin{align*}
		\Pr(& \ZKCont_{\relone}.\Verify(\crs, \bar{y}, X', \pi'_1, \relone) = 1 \ \wedge \ \ZKCont_{\relone}.\VerifyCom (\pp, X', \barx, b') = 1) ) \\
		& \ \ \ \ \ \ \ \ \ \ \ \ \ \ \ \ \ \ \ \ \ \Rightarrow \ \nizkR.\Verify(\crsR, X, \barz, \pi_2) =1 \ | \  \\
		& (\crs, \tw, \pp) \leftarrow \ZKCont_{\relone}.\Setup (1^{\lambda}, \relone), (\crstwo, \twtwo) \leftarrow \nizktwo.\Setup(1^{\lambda}), \\
		& (\bary, X', \pi'_1, b') \leftarrow A(\crs,\relone), (X, \pi_1, b) \leftarrow \ZKCont_{\relone}.\Reprove(\crs, X', \pi'_1, b', \relone), \\
		& \pi_2 \leftarrow \nizktwo.\Prove(\crstwo, X, \barz, \barx, b, \baromega_2)) = 1
\end{align*}}{
	\begin{footnotesize}
		\begin{align*}
			\Pr[& (\ZKCont_{\relone}.\Verify(\crs, \bar{y}, X', \pi'_1) = 1  \wedge \ZKCont_{\relone}.\VerifyCom (\pp, X', \barx, b') = 1 ) \\
			& \Rightarrow \ \nizkR.\Verify(\crsR, X, \barz, \pi_2) =1 \ | \ (\crs, \tw, \pp) \leftarrow \ZKCont_{\relone}.\Setup (1^{\lambda}, \relone), \\
			& (\crstwo, \twtwo) \leftarrow \nizktwo.\Setup(1^{\lambda}), (\bary, X', \pi'_1, b') \leftarrow \adv(\crs,\relone), \\ 
			& (X, \pi_1, b) \leftarrow \ZKCont_{\relone}.\Reprove(\crs, X', \pi'_1, b'), \\
			&\pi_2 \leftarrow \nizktwo.\Prove(\crstwo, X, \barz, \barx, b, \baromega_2)] = 1
		\end{align*}
\end{footnotesize}}

\begin{lemma}[Special Perfect Completeness] 
	\label{le:specialCompl_for_nizkR}
	If $\ZKCont_{\relone}$ is a zk continuation for $\relone$ and $\nizktwo$ is a NIZK for $\reltwo'$ for some appropriately chosen public parameters $\pp$, 
	then the $\nizkR$ construction described above has special perfect completeness.
\end{lemma} 
\begin{proof} This is easy to infer by combining the perfect completeness properties of $\nizktwo$ axnd perfect completeness 
	for $\ZKCont_{\relone}.\Reprove$.
	
	%  $\ZKCont_{\relone}.\Verify(\crs, \bar{y}, X', \pi'_1, \relone) = 1 \ \Rightarrow \ \ZKCont_{\relone}.\Verify(\crs, \bar{y}, X, \pi_1, \relone) = 1$
	%  $\ZKCont_{\relone}.\VerifyCom (\pp, X', \barx, b') = 1 \ \Rightarrow \ \ZKCont_{\relone}.\VerifyCom(\pp, X, \barx, b) = 1$
	%  $\pi_2 \leftarrow \nizktwo.\Prove(\crstwo, X, \barz, \barx, b, \baromega_2)) \ \Rightarrow \  \nizktwo.\Verify(\crstwo, X, \barz, \pi_2) =1$
\end{proof}

\noindent Finally, we define the notion of perfect zero-knowledge after reproving. \\
\begin{comment} 
	\noindent \textbf{Perfect Zero-knowledge after Reproving a $\ZKCont_{\relone}$ Proof} For all $\lambda \in \Prhbb{N}$, 
	for every benign auxiliary input $aux$, 
	for all $\bary,\barx,\barz, \baromega_1, \baromega_2$ with $(\bary, \barx; \baromega_1) \in \relone$ and 
	$(\barz, \barx; \baromega_2) \in \reltwo$, for all $X,\pi_1,\pi_2, b$, for every adversary $A$ it holds:
	\begin{align*}
		\Pr(& A(\crs, \realaux, \pi'_1,\pi_2, X', \rel) = 1 \ | \\
		& \ (\crs, \tw, \pp) \leftarrow \ZKCont_{\relone}.\Setup (1^{\lambda}, \relone), \\
		& (\pi'_1, X', \_) \leftarrow \ZKCont_{\relone}.\Reprove (\crs, X, \pi, b, \relone), \\
		& \pi_2 \leftarrow \nizktwo.\Prove(\crstwo, X, \barz, \barx, b, \baromega_2), \\
		&  \ZKCont_{\relone}.\Verify(\crs, \bary, X, \pi, \relone) = 1 
		\wedge \VerifyCom(\pp, X, \bar{x}, b) = 1) =  \\
		= \Pr(& A(\crs, \realaux, \pi',\pi_2, X', \rel) = 1 \ | \\ 
		& \ (\crs, \tw, \pp) \leftarrow \ZKCont_{\relone}.\Setup (1^{\lambda}, \relone), \\ 
		& (\pi'_1, X',\pi_2) \leftarrow \nizkR.\Sim(\tw, \bary, \relone) \\ 
		&  \ZKCont_{\relone}.\Verify(\crs, \bary, X, \pi, \relone) = 1
		\wedge \VerifyCom(\pp, X, \bar{x}, b) = 1)
	\end{align*}
\end{comment} 

\noindent \textbf{Zero-knowledge after Reproving a $\ZKCont_{\relone}$ Proof} For all $\lambda \in \mathbb{N}$, for every benign auxiliary input $aux$, 
for all $\bary,\barx,\barz, \baromega_1, \baromega_2$ with $(\bary, \barx; \baromega_1) \in \relone$ and $(\barz, \barx; \baromega_2) \in \reltwo$, for all $X',\pi_1', b'$, for every adversary $A$ it holds:
\eprint{\begin{align*}
		| \Pr(& A(\crs, \realaux, \pi_1,\pi_2, X, \rel) = 1 \ | \ (\crs, \tw, \pp) \leftarrow \ZKCont_{\relone}.\Setup (1^{\lambda}, \relone), \\
		%& (\crspk, \crsvk, \pp) \leftarrow \ZKCont_{\relone}.\Gen(\crs, \relone), \\
		& (\pi_1, X, \_) \leftarrow \ZKCont_{\relone}.\Reprove (\crs, X', \pi_1', b', \relone), \pi_2 \leftarrow \nizktwo.\Prove(\crstwo, X, \barz, \barx, b, \baromega_2), \\
		& \ZKCont_{\relone}.\Verify(\crs, \bary, X', \pi'_1, \relone) = 1, \ZKCont_{\relone}.\VerifyCom(\pp, X', \barx', b') = 1)   \\
		- \Pr(& A(\crs, \realaux, \pi_1,\pi_2, X, \rel) = 1 \ | \ (\crs, \tw, \pp) \leftarrow \ZKCont_{\relone}.\Setup (1^{\lambda}, \relone), \\ 
		%& (\crspk, \crsvk, \pp) \leftarrow \ZKCont_{\relone}.\Gen(\crs, \relone), 
		& (\pi_1,\pi_2, X) \leftarrow \nizkR.\Sim(\tw, \bary, \relone), \ZKCont_{\relone}.\Verify(\crs, \bary, X', \pi'_1, \relone) = 1, \\
		& \ZKCont_{\relone}.\VerifyCom(\pp, X', \barx', b') = 1) | \leq \negl(\lambda)
\end{align*}}
{\begin{align*}
		| \Pr[& \adv(\crs, \realaux, \pi_1,\pi_2, X, \rel) = 1 \ | \ (\crs, \tw, \pp) \leftarrow \ZKCont_{\relone}.\Setup (1^{\lambda}, \relone), \\
		%& (\crspk, \crsvk, \pp) \leftarrow \ZKCont_{\relone}.\Gen(\crs, \relone), \\
		& (\pi_1, X, \_) \leftarrow \ZKCont_{\relone}.\Reprove (\crs, X', \pi_1', b'), \\
		&\pi_2 \leftarrow \nizktwo.\Prove(\crstwo, X, \barz, \barx, b, \baromega_2), \\
		& \ZKCont_{\relone}.\Verify(\crs, \bary, X', \pi'_1, \relone) = 1, \ZKCont_{\relone}.\VerifyCom(\pp, X', \barx', b') = 1]  \\
		- \Pr[& \adv(\crs, \realaux, \pi_1,\pi_2, X, \rel) = 1 \ | \ (\crs, \tw, \pp) \leftarrow \ZKCont_{\relone}.\Setup (1^{\lambda}, \relone), \\ 
		%& (\crspk, \crsvk, \pp) \leftarrow \ZKCont_{\relone}.\Gen(\crs, \relone), 
		& (\pi_1,\pi_2, X) \leftarrow \nizkR.\Sim(\tw, \bary), \ZKCont_{\relone}.\Verify(\crs, \bary, X', \pi'_1) = 1, \\
		& \ZKCont_{\relone}.\VerifyCom(\pp, X', \barx', b') = 1]| \leq \negl(\lambda)
\end{align*}}

\begin{lemma}[ZK after Reproving a $\ZKCont_{\relone}$ Proof] 
	\label{le:specialZK_for_nizkR}
	If $\ZKCont_{\relone}$ is a zk continuation for $\relone$ and $\nizktwo$ is a NIZK for $\reltwo'$ for some appropriately chosen public parameters $\pp$, 
	then the $\nizkR$ construction described above has zero-knowledge after reproving a $\ZKCont_{\relone}$ proof.
\end{lemma} 
\begin{proof} The statement follows from the perfect zero-knowledge w.r.t. $\relone$ for $\ZKCont_{\relone}$ and 
	the zero-knowledge property of $\nizktwo$ w.r.t. $\reltwo'$.
\end{proof}

