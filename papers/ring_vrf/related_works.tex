% !TEX root = crypto.tex

\subsection{Related Work}

\noindent\textit{Ring Signatures.} \liz{Mention standard techniques for proving membership in the ring.}
\\\\
\noindent\textit{Unique Ring Signatures.} The unique ring signature framework \cite{URCframework} is the closest model to our ring VRF framework particularly in terms of the presence of a deterministic component known as the unique identifier for the signed message. This identifier remains constant for the same signed message even when the ring changes.  Essentially, the unique identifier in the unique ring signature model and the ring VRF evaluation value function equivalently in both models. However, a fundamental distinction lies in the treatment of this identifier. In our ring VRF model, we impose the requirement of pseudorandomness, as defined in \cite{ucvrf,praos}, on this unique identifier, even in the case of malicious parties.
This requirement is crucial for applications such as lotteries or leader elections where the unique identifier plays a privileged or reward-based role based on predefined conditions. Another definitional difference is that a ring VRF  signature not only prove the correctness of the evaluation value of an input but also signs an auxiliary data independent from the input. This property is needed for anonymous access mechanisms to prevent replay attacks because auxiliary data can be used to effectively bind the ring VRF signature to e.g., a TLS session.
The signature size of unique ring signature schemes scales either linearly \cite{URCframework,URCfc} or logarithmically  \cite{URCblockchainprivacy,URClattice} with the size of the ring. In contrast, our ring VRF constructions maintain a constant signature size while providing stronger security guarantees. Also our signing and verification algorithms show better asymptotic scalability compared to existing unique ring signatures because they operate with a constant-size  commitment to the ring.

Unique ring signature (URS) schemes \cite{URCframework} aim to address similar challenges as ring VRF in the context of anonymous identity applications. Both generate a unique identifier within the ring signature for each input, which corresponds to the ring VRF output in our case. Unlike ring VRF, where a party can sign any message with a ring VRF signature, unique ring signature schemes do not include the capability to sign such messages. Therefore, leveraging these identifiers for practical authentication, such as in a TLS session, is not straightforward. 
Beyond this, we demand from a ring VRF output  to be a pseudorandom even if the signer's key is maliciously generated. This property distinguishes it from unique ring signatures. 
\liz{EC Reviewer A: ``The tag of a unique ring signature has to implicitly be pseudorandom for honest keys in order to not break anonymity. Then, since the tag is required to be unique, it will likely be pseudo-random for maliciously generated keys as well."}
Although this property may not find immediate use in the identity applications we mentioned, it holds critical significance in applications that grant privileges to parties based on specific criteria associated with their ring VRF output, such as leader elections or lotteries. 
Ring VRF provides the pseudorandomness property required in these leader election mechanism.
\\\\
\noindent\textit{Linkable and Traceable Ring Signatures.} Other related models are linkable ring signatures \cite{ring_linkable,ring_linkablee} and traceable ring signatures \cite{traceable07,traceable_sub}. Linkable ring signatures allows a third party to link whether two ring signatures of two inputs are signed by the same party in the same ring without revealing the identity. This type of linkability property is valuable in applications that impose restrictions on authentications, such as preventing double spending or multiple voting. Akin to ring VRF and unique ring signatures, if a signer signs the same message twice for the same ring and issuer, it becomes evident that both signatures are produced by the same party, although the specific party's identity remains secret. \liz{Is linkability formally capture in our UC definition?} Both ring VRF and unique ring signature schemes have this property in a single context through the unique identifier for each party.
Differently than ring VRF, traceable ring signatures  disclose the identity of the signer when the signer generates two signatures for two different inputs within the same ring and from the same issuer.
\\\\
\noindent\textit{Verifiable Random Functions.} \liz{Talk about Unbiasable VRF paper.}
\\\\
\noindent\textit{Ring VRFs in Practice.} 
Ring VRFs are of theoretical and practical interest and are already being deployed.
Indeed, Semaphore \cite{Semaphore} is similar to a ring VRF in our formalism, which provides a ``nullifier," unique per identity and context but anonymous, along with a signature on a message. However, no formal security definition is provided for Semaphore.  Moreover, our construction distinguishes itself by offering more efficient proving times and offers the potential for proof reuse.

Sassafras~\cite{Sassafras} is a constant-time block production protocol, which ensures that there is exactly one block produced with constant-time intervals, rather than multiple or no blocks at all.
Each validator anonymously submits a VRF output (or several VRF outputs). These then determine the order of block producers in the next epoch, when a validator can post a VRF proof to show that it was their output in the corresponding slot. The question then becomes how to anonymously submit a VRF output. The ring VRF gives a proof that the output came from some validator. The protocol also needs an anonymous submission mechanism: it suggests just sending to a random validator who forwards it to everyone.
\liz{What to say about what our paper is doing here...}
\\\\
\noindent\textit{Anonymous VRF.} \liz{https://eprint.iacr.org/2018/1105 introduces the notion of an anonymous VRF and provides some other applications, like voting, which may help further motivate Ring VRF as well. However, it seems AVRF falls short for other applications, like decentralized identity, described as follows.} An anonymous VRF \cite{anonymousVRF} is a special type of VRF designed to enable verification of the VRF output without dependence on the party's key.  Differently than ring VRF, the verification is executed with another public key which is generated from the public key of the party.  A crucial distinction lies in their  uniqueness definitions, as anonymous VRFs ensure the uniqueness of VRF outputs for each (updated) public key and input. Consequently, anonymous VRFs are not suitable for identity applications where the VRF output serves as a unique and anonymous identifier, as each updated public key generates a different VRF output.
Another notable difference is related to the pseudorandomness definition, which does not guarantee pseudorandomness even when the key belongs to a malicious party. This limitation can pose challenges in applications like consensus mechanisms as described in \cite{anonymousVRF}, making their use potentially infeasible.
\\\\
\noindent\textit{Concurrent Work.} \liz{https://eprint.iacr.org/2022/1290 introduces the notion of a ring VRF and constructs it from ECVRF + One-Out-of-Many-Proofs. How does their definition compare? Their construction does not double hash the output, which I thought was necessary for (UC) security. Can we find something wrong with this paper?}
\\\\
\noindent\textit{Commit-and-Prove SNARKs.} zk-continuations are an example of the commit-and-prove approach \cite{LegoSNARK}, linking  in a way similar to the ccGroth16 construction from LegoSNARK \cite{LegoSNARK}. Our work extends this concept by formalizing the reuse of previously generated proofs through simple transformations while maintaining the zero-knowledge property. Our protocol $ \SpecialG $ is very similar to the ccGroth16 construction from LegoSNARK \cite{LegoSNARK} with the additional feature of providing an interface for rerandomizing previously generated proofs, all while preserving the zero-knowledge property. \liz{How do ZK continuations differ from malleable proof systems with strong derivation privacy?}
\\\\
 \emph{Anonymous Credentials.} In the identity application given above, ring VRFs resemble context-specific pseudonyms, which first appeared in the context of anonymous credentials~\cite{DBLP:journals/cacm/Chaum85,DBLP:conf/crypto/ChaumE86,EC:CamLys01}. \liz{Comparison with ring VRF? Cite DAA papers too.}
% In this setting, users have distinct and unlinkable pseudonyms across organizations who play an active role in engaging with users transferring their credentials between them. Ring VRFs enable organizations to be entirely passive, each one defining their own set of context strings and otherwise allowing users to create and authorize their pseudonym for the contexts that are of interest to them.
\\\\
In \cite{DBLP:conf/eurocrypt/DamgardDP06,DBLP:conf/ccs/CamenischHKLM06}, users are unlinkable across discrete time periods, which is a more limited setting compared to arbitrary context strings.