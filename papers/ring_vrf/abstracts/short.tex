% !TEX root = ../crypto.tex


\def\eprintsmallskip{\smallskip}{}%
%1. VRFs allow a user to...
%2. VRFs are important in PoS (ECVRF standard)
%4. To have a degree of anonymity, we define a new primitive, ring VRF
%5. UC realize it
%6 based on ECVRF (standard)
%, and applications to identity
A verifiable random function (VRF) is a pseudorandom function (PRF) whose output is publicly verifiable.
%That is, for a given input, the output is coupled with a proof of correct evaluation.
Verifiable random functions are a key ingredient in many proof-of-stake blockchains for ensuring fair and verifiable pseudorandom lotteries.
For example, in Single Secret Leader Election (SSLE), only one leader is selected per slot, and the
leader remains anonymous until they announce themselves by providing a proof.
Anonymity of the leader is an important security feature, as the leader makes for an attractive target and single point of failure.
The main security properties of SSLE are uniqueness (i.e., electing only one leader), unpredictability (i.e., hiding the leader), and fairness (i.e., having equal chance of becoming the leader).
%In addition to SSLE, ring VRFs are an attractive option for deployment in decentralized identity (DID) systems.
%Ring VRF finds applications in a wide range of other cases, including rate limiting and rationing systems.

~~~~~~~With the above properties in mind, we introduce a new cryptographic primitive, called a 
\emph{ring verifiable random function.}
Ring VRFs enjoy features of both verifiable random functions and ring signatures, namely unique, pseudorandom, verifiable outputs as well as anonymity of a signer within a ring of potential signers.
In particular, ring VRFs fulfill the \emph{unique}, \emph{pseudorandom}, \emph{anonymous} selection of a leader.
%We design its security in the universal composability (UC) framework and construct two protocols secure in our model.
% Liz: removing this for now since it doesn't appear relevant to Sassafras
%We also formalize a new notion of \emph{zero-knowledge (ZK) continuations} allowing for the reusability of proofs by randomizing and enhancing the efficiency of one of our ring VRF schemes. We instantiate this notion with our protocol $ \SpecialG $ which allows a prover to reprove a statement in a constant time and be unlikable to the previous proof(s). 
We provide a formal security definition for ring VRFs in the Universal Composability (UC) framework.
We then propose an efficient instantiation of a ring VRF and prove that it realizes our functionality.
In particular, we build a ring variant of ECVRF, an elliptic-curve-based VRF currently undergoing standardization through the IRTF.
We further optimize our construction by employing a \emph{zero-knowledge (ZK) continuation}, which enables reproving a statement in a constant time in a manner that is unlikable to previous proofs.
A ZK continuation allows for the reusability of proofs of ring membership, but is likely of independent interest.
%which allows a prover to reprove a statement in a constant time and be unlikable to the previous proof(s). 
%and which has been shown to achieve strong notions of security (cite Unbiasable VRF paper)
In addition to SSLE, ring VRFs are an attractive option for identity systems, where users can register their public keys and generate pseudonyms using ring VRF outputs, protecting user privacy while preventing Sybil attacks.
%\liz{Emphasize privacy-enhancing technologies if we are submitting to SAC.} 

% OLD
%We introduce a new cryptographic primitive,  named
%\emph{ring verifiable random function (ring VRF)}. Ring VRF combines properties of VRF  and ring signatures, offering verifiable unique, pseudorandom outputs while ensuring  anonymity of the output and message authentication. We design its security in the universal composability (UC) framework and construct two protocols secure in our model.
%We also formalize a new notion of \emph{zero-knowledge (ZK) continuations} allowing for the reusability of proofs by randomizing and enhancing the efficiency of one of our ring VRF schemes. We instantiate this notion with our protocol $ \SpecialG $
%which allows a prover to reprove a statement in a constant time and be unlikable to the previous proof(s). 

%ty to sign a message 
%% which enables better anonymous credentials...
%% Anonymized
%%\eprint{Ring VRFs are}{We introduce ring VRFs, which are}
%Ring VRF is a ring signature that proves a correct evaluation
%of a random, while hiding the signer's
%identity within a ring, some set of possible signers. We design a ring VRF protocol which has efficient instantiations with our novel {\em zero-knowledge continuation} technique.
%% \eprint{We propose ring VRFs as a natural fulcrum around which a diverse array of zkSNARK circuits turn, making them an ideal target for optimization and eventually standards.}{}
%We demonstrate a {\em zero-knowledge continuation} technique,
%which works by adjusting a Groth16 trusted setup to hide public inputs
%when rerandomizing the Groth16, ensuring that muliple uses of a proof generated once are unlinkable.  We then build ring VRFs that amortizes
%expensive ring membership proofs across many ring VRF signatures.
%%
%Our ring VRF needs only eight $\mathcal{G}_1$ and two
%$\mathcal{G}_2$ scalar multiplications, making it the only ring signature
%with performance competitive with group signatures.
%
%A ring VRF can be used to obtain a unique pseudo-anonymous identity from a given a list of identities.
%By using a different input for the ring VRF in different contexts, we can generate a pseudonym for each context that is unlinkable between different contexts. 
%We discuss applications that range across the anonymous credential space.

%Ring VRFs produce a unique identity for any given context but remain
%unlinkable between different contexts.  These unlinkable but unique
%pseudonyms provide a far better balance between user privacy and service
%provider or social interests than attribute based credentials like IRMA credentials.
%Ring VRFs also support anonymously rationing or rate limiting resource
%consumption that winds up vastly more flexible and efficient than
%purchases via money-like protocols.

%We define the security of ring VRFs in the universally composable (UC) model and show that our protocol is UC secure.
