% !TEX root = crypto.tex

%\newcommand{\handan}[2]{{\sout{#1}}\textcolor{blue}{#2}}
%\newcommand{\handan}[2]{{#1}{}}

%\newcommand{\qhandan}[2]{{\uline{#1}}\textcolor{pink}{(#2)}}
%TO REMOVE ALL QUESTIONS
%\newcommand{\qhandan}[2]{{#1}{}}

\section{Zero-knowledge Continuations}
\label{sec:rvrf_cont}
In this section, we describe our new notion that we call a zero-knowledge (ZK) continuation. Our new notion focuses on optimizing a NIZK proving system tailored for a relation $ \rel $ such that
$$\rel = \{(\bary, \barz; \barx, \baromega_1, \baromega_2):  (\bary, \barx; \baromega_1) \in \relone, (\barz, \barx; \baromega_2) \in \reltwo \},$$
and $\relone$, $\reltwo$ are NP relations. 
At a high level, NIZK proving systems for relations as $ \rel$ are based on the commit-and-prove methodology~\cite{Kilian1990UsesOR,CLOS02,LegoSNARK} 
as relations $\relone$ and $\reltwo$ have input $\barx$ in common. These systems typically incorporate a commitment $X$ to $\barx$ in their respective 
proofs or arguments for $ \relone $ and $ \reltwo $ to hide the witness $ x $ in $ \rel $. In our proposed NIZK  for $ \rel $, we adopt a similar methodology but with a distinctive addition. Our design is specified to facilitate the efficient re-proving membership for relation $\relone$ via ZK continuation. 
In practice, using a NIZK that ensures a ZK continuation for a 
subcomponent relation (i.e., in our case $\relone$) means one essentially needs to create  an  expensive proof for that subcomponent 
relation only once; the initial proof can later be re-used multiple times (just after inexpensive operations), 
while preserving knowledge soundness and zero-knowledge of the entire NIZK. Thus, our re-used proofs stay unlinkable. 
Below, we formally define ZK continuation. In Section~\ref{sec:rvrf_groth16}, we instantiate it via  \SpecialG, and finally, in Section~\ref{subsec:rvrf_faster}, we use it to instantiate our $ \rVRF.\Sign $ algorithm from Section \ref{sec:pedersen_vrf}\eprint{ with fast amortised prover time}{}. \\

%\noindent In addition, the anonymity property of our ring VRF demands we not only finalise multiple times a component of the zero-knowledge 
%continuation but also each time the result remains unlinkable to previous finalisations, meaning our ring VRF stays zero-knowledge 
%even with a continuation component being reused. We formalise such a more general zero-knowledge property in 
%section~\ref{sec:rvrf_groth16} and give an instantiation of our NIZK fulfilling such a property in section~\ref{subsec:rvrf_faster}. 
%Moreover, the anonymity property of a ring VRF demands we finalise the amortized ``continuation'' multiple
%times, with each time being unlinkable to the others, meaning our rVRF
%stays zero-knowledge even with the continuation being reused.


%\begin{definition}[ZK Continuations] A zero-knowledge continuation $\SpecialG_\rel$ consists of four algorithms 
%($\SpecialG_\rel.\Preprove$, $\SpecialG_\rel.\Reprove$, $\ldots$, $\SpecialG_\rel.\Verify$) such that:
%\begin{itemize}
%\item $\SpecialG_\rel.\Preprove : (\bar{y}, \bar{x}; \baromega_1) \mapsto (X,\pi)$ \,
%constructs a commitment $X$ and a proof $\pi$ for relation $\rel$ from a vector 
%of inputs $\bar{y}$ (called \em{transparent}), a vector of inputs $\bar{x}$ (called \em{opaque}), and witnesses $\baromega_1$.
%\item $\SpecialG_\rel.\Reprove : (X,\pi) \mapsto ((X',\pi'); b)$ \,
%finalises the commitment $X'$ and proof $\pi'$ and returns an opening $b$ for the commitment. 
%\item $\SpecialG_\rel.\Verify(\bar{y}; (X',\pi') )$ \, 
%verifies the 
%\end{itemize}
%%TO DO: add an algorithm to the $\SpecialG_\rel$ such that: Our \Verify needs a separate proof-of-knowledge for $X'$, 
%%the production of which requires knowledge of $\bar{x}$, and occurs in parallel to \Reprove.

%We define (white-box) knowledge soundness for zero-knowledge continuations
%exactly like for zero-knowledge proofs, but with the composition 
%$\Prove : (\bar{y}, \bar{x}; \baromega_1) \mapsto \Reprove(\Preprove(\bar{y}, \bar{x}; \baromega_1))[0]$
%as well as this additional proof-of-knowledge.
%Zero-knowledge however should hold even if \Reprove gets invoked multiple
%times upon the same \Preprove results, again even with the additional proof-of-knowledge.
%\end{definition} 

%TODO: Check notation consistancy with NIZK in preliminaries
\begin{definition}[ZK Continuation]
\label{def:zk_cont}
A ZK continuation for a relation $\relone$ with a vector of 
inputs $(\bary, \barx)$ and witnesses $\baromega_1$ is a tuple of PPT algorithms 
$(\ZKCont_{\relone}.\Setup, \allowbreak \ZKCont_{\relone}.\Preprove, \ZKCont_{\relone}.\Reprove, 
\ZKCont_{\relone}.\VerifyCom, \allowbreak \ZKCont_{\relone}.\Verify, \allowbreak \ZKCont_{\relone}.\Sim)$
with  implicit inputs $ \rel_1  $ and security parameter $\lambda$:
\begin{itemize}

\item $\ZKCont_{\relone}.\Setup(1^{\lambda}) \rightarrow (\crs, \tw, \pp):$ It outputs a common reference 
string $\crs$, a trapdoor $\tw$ and a list $\pp$ of public parameters.

\item $\ZKCont_{\relone}.\Preprove(\crs, \bar{y}, \bar{x}, \baromega_1) \rightarrow (X', \pi', b'):$ It
outputs a commitment $X'$ to $\bar{x}$ (called \emph{opaque}), its opening $ b' $ and a proof $\pi'$ constructed from vector of inputs $\bar{y}$ (called \emph{transparent}).
\item $\ZKCont_{\relone}.\Reprove(\crs, X', \pi', b') \rightarrow (X, \pi, b):$ It
outputs a new commitment $X$ and proof $\pi$ with a new opening $b$ for the commitment. 

\item $\ZKCont_{\relone}.\VerifyCom(\pp, X, \barx, b) \rightarrow 0/1:$
It verifies that  $X$ is a commitment to $\barx$ with opening $b$ and 
outputs 1 if indeed that is the case and 0 otherwise.
 
\item $\ZKCont_{\relone}.\Verify(\crs, \bar{y}, X, \pi) \rightarrow 0/1:$ It outputs $1$ if it verifies  and $0$ otherwise.

\item $\ZKCont_{\relone}.\Sim(\tw, \bary)\rightarrow (\pi, X):$ It outputs a proof $\pi$ and $X$ given a simulation trapdoor $\tw$ and statement $(\bary, \barx)$. 

\end{itemize}
$ \ZKCont_{\relone} $ satisfies perfect completeness, knowledge soundness and zero-knowledge as defined below.
We define perfect completeness for $\Preprove$ and $\Reprove$  separately  in the most general way possible
(i.e., with inputs supplied by the adversary where possible).  
\\\\
\noindent \textbf{Perfect Completeness for $\Preprove$:} For all $\lambda $, for every $(\bary, \barx; \baromega_1) \in \relone$:
\begin{footnotesize}
\begin{align*}\
\Pr[&\ZKCont_{\relone}.\Verify(\crs, \bar{y}, X, \pi) = 1 \wedge  \ZKCont_{\relone}.\VerifyCom (\pp, X, \barx, b) = 1\  | \ \\ 
&(\crs, \tw, \pp) \leftarrow \ZKCont_{\relone}.\Setup (1^{\lambda}), (X, \pi, b) \leftarrow \ZKCont_{\relone}.\Preprove(\crs, \bar{y}, \bar{x}, \baromega_1)] = 1
\end{align*}
	
\end{footnotesize}

\noindent \textbf{Perfect Completeness for $\Reprove$:} For all $\lambda$ and PPT adversaries $\adv$: 
\begin{footnotesize}
\begin{align*}
\Pr [& (\ZKCont_{\relone}.\Verify(\crs, \bar{y}, X', \pi') = 1 \  \Rightarrow \ZKCont_{\relone}.\Verify(\crs, \bar{y}, X, \pi) = 1)  \ \wedge \ \\
                   &\wedge \ (\ZKCont_{\relone}.\VerifyCom (\pp, X', \barx, b') = 1  \Rightarrow  \ZKCont_{\relone}.\VerifyCom (\pp, X, \barx, b) = 1) \ | \\
                   & (\crs, \tw, \pp) \leftarrow \ZKCont_{\relone}.\Setup (1^{\lambda}),  (\bary, \barx, X', \pi', b') \leftarrow \adv(\crs,\relone), \\
                   & (X, \pi, b) \leftarrow \ZKCont_{\relone}.\Reprove(\crs, X', \pi', b')] = 1
\end{align*}
\end{footnotesize}

\noindent \textbf{Knowledge Soundness:} For all $\lambda$, for every benign auxiliary input $\realaux$ (as per~\cite{bening_auxiliary}) and 
every non-uniform efficient adversary $\adv$, there exists an efficient non-uniform extractor  $ \Extract $ such that:
\begin{footnotesize}
\begin{align*}
\Pr [& \ZKCont_{\relone}.\Verify(\crs, \bar{y}, X, \pi) = 1  \wedge \ZKCont_{\relone}.\VerifyCom(\pp, X, \bar{x}, b) = 1  \wedge (\bary, \barx; \baromega_1) \notin \relone\ | \\
                   & (\crs, \tw, \pp) \leftarrow \ZKCont_{\relone}.\Setup (1^{\lambda}), (\bary, \barx, X, \pi, b; \baromega_1 ) \leftarrow \adv || \Extract(\crs, \realaux, \relone)] = \negl (\lambda),
\end{align*}
\end{footnotesize}

\noindent where by $(\mathit{output_{A}};\mathit{output_{B}}) \leftarrow A || B(\mathit{input})$ we denote algorithms $A$, $B$ running on the same 
$\mathit{input}$ and $B$ having access to the random coins of $A$. \\

 \emph{Finally, we introduce a new flavor of zero-knowledge property for $ \ZKCont_{\relone} $. It allows us to formalize the concept that after an initial call to $\ZKCont_{\relone}.\Preprove$ with $((\bary, \barx), \baromega_1) \in \relone$, subsequent sequential calls to $ \ZKCont_{\relone}.\Reprove $ result in proofs that disclose no information about $\barx$ or $\baromega_1$.
Hence, the proofs obtained via sequential use of $ \ZKCont_{\relone}.\Reprove $ as described above are not linkable, i.e., a property targeted  in the preamble of this section. }
\liz{Where?}
\\\\
\noindent \textbf{Perfect Zero-knowledge w.r.t. $\relone$:} For all $\lambda$, there exists a PPT algorithm $ \Sim $ such that for all  $(\bary, \barx; \baromega_1) \in \relone$, for every benign auxiliary input $\realaux$, for all $X'$, for all $\pi'$, for all $b'$ , for every adversary $\adv$:
\begin{footnotesize}
\begin{align*}
& \Pr [\adv(\crs, \realaux, \pi, X, \relone) = 1 \ | \ (\crs, \tw, \pp) \leftarrow \ZKCont_{\relone}.\Setup (1^{\lambda}), 
\\& \ZKCont_{\relone}.\Verify(\crs, \bary, X', \pi') = 1,  (\pi, X, \_) \leftarrow \ZKCont_{\relone}.\Reprove (\crs, X', \pi', b')] \\
= & \Pr[A(\crs, \realaux, \pi, X, \relone) = 1 \ | \ (\crs, \tw, \pp) \leftarrow \ZKCont_{\relone}.\Setup (1^{\lambda}), \\ 
                        & \ZKCont_{\relone}.\Verify(\crs, \bary, X', \pi') = 1,  (\pi, X) \leftarrow \ZKCont_{\relone}.\Sim(\tw, \bary)]
\end{align*}

\end{footnotesize}


\end{definition} 



\subsection{Specialized Groth16 Proofs}
\label{sec:rvrf_groth16}

In this section, we instantiate our ZK continuation notion with a scheme that we call  \emph{$\mathsf{SpecialG}$}. It is based on the Groth16 zkSNARK~\cite{Groth16}.
As in~\cite{Groth16}, we use a standard quadratic arithmetic program $ \cQ $ (QAP) of size $ m $ defined over field $ \F_q $. 
(See the full version of the paper at this anonymous link for more details~\cite{anonymous}.)
%(See Appendix~\ref{appendix_specialg} for more details.) 
Then given $ \cQ $, we set  $ \relRQ  $ that corresponds to $ \relone $. $ \relRQ $  consists of pairs $((\bary, \barx); \baromega) \in \F_q^{l} \times \F_q^{n-l} \times \F_q^{m-n}$ where $\F_q$ is a field.  

%\eprint{\begin{definition}[QAP] 
%\label{def:QAP}
%A Quadratic Arithmetic Program (QAP) $\cQ = (\cA, \cB, \cC, t(X))$ of size $m$ 
%and degree $d$ over a finite field $\F_q$ is defined by three sets of polynomials $\cA = \{a_i(X)\}_{i=0}^m$, 
%$\cB = \{b_i(X)\}_{i=0}^m$, $\cC = \{c_i(X)\}_{i=0}^m$, each of degree less than $d-1$ and a target degree $d$ 
%polynomial $t(X)$. Given $\cQ$ we define $\relRQ$ as the set of pairs 
%$((\bary, \barx); \baromega) \in \F_q^{l} \times \F_q^{n-l} \times \F_q^{m-n}$ for which it 
%holds that there exist a polynomial $h(X)$ of degree at most $d-2$ such that:
%$$(\sum_{k=0}^m v_k \cdot a_k(X)) \cdot (\sum_{k=0}^m v_k \cdot b_k(X)) = (\sum_{k=0}^m v_k \cdot c_k(X)) + h(X)t(X) \ \ (\ast)$$ 
%where $\barv = (v_0, \ldots, v_m) = (1, x_1, \ldots, x_n, w_1, \ldots w_{m-n})$ and $\bary = (x_1, \ldots, x_l)$ and 
%$\barx = (x_{l+1}, \ldots, x_n)$ and $\baromega = (w_1, \ldots, w_{m-n})$. 
%\end{definition}}{}


%SUMMARY%%%%
%\noindent In summary, \SpecialG setup is an extension of original Groth16~\cite{Groth16} setup by two additional 
%group elements. $\SpecialG$ setup is identical to commit-carrying LegoSNARK 
%$\ccgroth$~\cite[Fig.~22]{LegoSNARK} setup\footnote{$\ccgroth$ is, in turn, based on  Groth16.}.
%Moreover, together the $\Preprove$ and the $\Reprove$ procedures of $\SpecialG$ are identical to the proving  
%procedure in $\ccgroth$ with the difference that $\SpecialG$ also re-randomises the commitment to part 
%of the public input, as part of  $\Reprove$. Additionally, the verification procedure and the commitment verification are identical to their 
%counterparts in $\ccgroth$. \eprint{}{We give $\SpecialG$ in full in Appendix~\ref{appendix_specialg} and below we only detail:} \\ 
%%%%%%%%%%%

%Second, our $\SpecialG.\Reprove$ algorithm uses a Groth16 re-randomisation 
%technique for the proof (see~\cite[Fig.~1]{RandomizationGroth16} or LegoSNARK~\cite[Fig.~22]{LegoSNARK}), 
%but, in addition, $\SpecialG.\Reprove$ also re-randomises $X$ which is a commitment to a slice of the public input;
%Moreover, in terms of security properties, we appropriately define the zero-knowledge for zk continuations such 
%that even after iteratively applying $\SpecialG.\Reprove$ zero-knowledge property is preserved for both the witness 
%as well as the public input committed to in $X$.  

\noindent We let $\ecE$ be a pairing-friendly elliptic curve with an efficient and 
non-degenerate pairing $e$. We denote its first  and second source groups by $ \grone,\grtwo $ with generators 
$\gone$ and $\gtwo$, respectively.
%Let $\mathbb{F}_q$ be a prime field, 
%let $G_1$, $G_2$, $G_T$ be as defined in section~\ref{??}, let $e$, $g$, $h$ be defined as $\ldots$. Let $t(X)$ and
%$\{u_i(X),v_i(X),w_i(X)\}_{i=0}^m$ be polynomials in $\F_q[X]$, let $\ldots$ be $\ldots$ such that there exists $h(X) \in \F_q[X]$ with
% $$ \sum_{i=0}^m a_iu_i(X) \cdot  \sum_{i=0}^m a_iv_i(X)  = \sum_{i=0}^m a_iw_i(X) + h(X)t(X)  \ (\ast)$$
%Then let $\relone = \{ (;) \ | \ (;)  (\ast) \}$
Given a vector $ \vec{x} $ of field elements  and a group element $ G \in \grone $ or $ \grtwo $, we use short hand notation $ \vec{x} \cdot G $ to naturally represent the corresponding vector of group elements.
$ \SpecialG $ for relation $\relRQ$ works as follows: %instantiation of the zero-knowledge continuation notion from Definition~\ref{def:zk_cont}:
\begin{itemize}
\item $\SpecialG.\Setup(1^{\lambda}, \relRQ) \rightarrow (\crs, \tw,\pp)$:  It is identical to the LegoSNARK  $\ccgroth$~\cite[Fig.~22]{LegoSNARK} setup which  is  an extension of original Groth16~\cite{Groth16} setup by two additional group elements in $ crs $ and one field element in $ \tw$ which are underlined next. $\tw $ consists of $ \alpha, \beta, \gamma, \delta, \tau, \underline{\eta} \leftsample \F_q^*$ and $\crs = (\barsig_1, \barsig_2)$ where 
\begin{align*}
\barsig_1 &= (\alpha  , \beta  , \delta  , \{\tau_i  \}_{i=0}^{d-1}, \{\gamma_i\}_{i=1}^n, \underline{\frac{\eta}{\gamma}}, \{\delta_i\}_{i=n+1}^m, \{\frac{1}{\delta}\tau^it(\sigma) \}_{i=0}^{d-2}, \underline{\frac{\eta}{\delta}})\cdot \gone \\
\barsig_2 &= (\beta  , \gamma  , \delta  , \{\tau^i  \}_{i=0}^{d-1})\cdot \gtwo
\end{align*}
Here, $ \gamma_i = \frac{\beta a_i(\tau)+ \alpha b_i(\tau)+ c_i(\tau)}{\gamma} $ and $ \delta_i =\frac{\beta a_i(\tau)+ \alpha b_i(\tau)+ c_i(\tau)}{\delta} \}_{i=n+1}^m  $.
%\item $\SpecialG.\Gen: (\crs, \relRQ) \mapsto (\pp, \crspk, \crsvk)$ where \\
%$\crspk = \left([\barsig_1]_1, [\beta]_2, [\delta]_2, \left\{[\tau^i]_2\right\}_{i=0}^{d-1}\right)$  \\ 
%$\crsvk = \left([\alpha]_1, \left\{ \left[ \frac{\beta a_i(\tau)+ \alpha b_i(\tau)+ c_i(\tau)}{\gamma} \right]_1 \right\}_{i=1}^{l}, 
%[\beta]_2, [\gamma]_2, [\delta]_2\right)$  \\ 
We let $\pp = (\{K_i\}_{\ell+1}^n,K_\gamma) = (\{\gamma_i\gone\}_{i=l+1}^{n},   \frac{\eta}{\gamma}\gone )$ and $\Kdelta = \frac{\eta}{\delta} \cdot \gone$.

A Groth16 proof for $ \relRQ $ needs the public statement $ (\bary,\barx)$ for verification. Differently than this, we want to achieve the verification of a Groth16 proof without $ \barx $ but with the commitment to $ \barx $. Therefore, we need additional elements in $ \crs $ to be able to still execute the verification.
%{We should say what is the difference from Groth16.Setup or it is the same. I think in general in $ \SpecialG $, you should tell which part is from Groth16 or Legosnark and where we %change it while describing the algorithms. It will be much clear for the reader to verify. You have notes in the end but I think it is better to have it while describing since you can tell %more right away from the algorithm than in the end of everything}
\item $\SpecialG.\Preprove(\crs, \bar{y}, \bar{x}, \baromega_1) \leftarrow (X', \pi', b')$: It runs the proving algorithm of the Groth16 SNARK outputting $ \pi' = (A, B, C)  \in (\grone, \grtwo, \grone)$. Then it lets $ b' = 0 $ and computes the deterministic commitment $ X' = \sum_{i = \ell +1}^n x_i K_i$ to  $ \barx $.
\eprint{In more detail, 
\begin{align*}
&\ \ \ \ \ \ \ \   b' = 0; r, s\xleftarrow{\$} \F_p; X' = \sum_{i=l+1}^{n} v_i \cdot  \frac{\beta a_i(\tau)+ \alpha b_i(\tau)+ c_i(\tau)}{\gamma} \cdot \gone;  \\
&\ \ \ \ \ \ \ \  a= \alpha + \sum_{i=0}^{m} v_i \cdot a_i(\tau) + r \cdot \delta; b = \beta + \sum_{i=0}^{m} v_i \cdot b_i(\tau) + s \cdot \delta; \\ 
&\ \ \ \ \ \ \ \  c = \frac{\sum_{i=n+1}^{m} (v_i (\beta a_i(\tau)+ \alpha b_i(\tau)+ c_i(\tau))) + h(\tau)t(\tau)}{\delta}  + o\cdot s + u \cdot r - r \cdot s \cdot \delta; \\
&\ \ \ \ \ \ \ \  \pi' = (A,B,C) =(a \gone, u\gtwo, v \gone), 
\end{align*}}{}
\eprint{\noindent where $\barv = (1, x_1, \ldots, x_n, w_1, \ldots, w_{m-n})$, $\bary = (x_1, \ldots, x_l)$, $\barx = (x_{l+1}, \ldots, x_n)$, 
$\baromega = (w_1, \ldots, w_{m-n})$ (same as per definition of QAP).}{}
\item$\SpecialG.\Reprove(\crs, X', \pi', b') \leftarrow (X, \pi, b):$ It rerandomizes $ \pi $ and the commitment $ X $  by using $ b' $, i.e., given $ \pi' = (A',B',C') $, pick $ b , r_1, r_2  \xleftarrow{\$} \F_p $ and let $ X = X' + (b- b') \Kgamma, \pi = (A, B, C)  $ where $ A = \frac{A'}{r_1}, B = r_1 B' + r_1r_2 \delta \gtwo, C = C' + r_2A'  - (b - b') \Kdelta $.


The $ \Preprove $ and $ \Reprove $ procedures of $ \SpecialG $ are identical to the proving procedure in $ \ccgroth $. In $ \SpecialG $, we split these procedures because we aim to run $ \Preprove $ once which contains heavier operations and then we can efficiently run $ \Reprove $ multiple times  with lighter operations.   

%The randomization process of $ \pi' $ aligns with the proving algorithm of $ \ccgroth $ \cite{LegoSNARK}. Notably, our approach involves a \emph{re}-randomization step using the blinding factor $ b' $ of $ \pi' $. In essence, whereas $ \ccgroth $ employs a new blinding factor $ b $ and generates a new proof $ \pi $ according to $ b $, we utilize $ b-b' $ instead and modify $ \pi' $ accordingly to generate a randomized proof $ \pi $. Actually, $ \SpecialG $ gives a way to obtain a new proof by randomizing another valid proof by leveraging the technique introduced in $ \ccgroth $. However, in the end, the resulting proofs of both schemes are  generated with the same operations. 
%\begin{align*}
%&\ \ \ \ \ \ \ \  b , r_1, r_2  \xleftarrow{\$} \F_p, X = X' + (b- b') \Kgamma, \pi = (O, U, V), \\
%&\ \ \ \ \ \ \ \ O = \frac{1}{r_1} O', U = r_1 U' + r_1r_2 \delta \gtwo, V = V' + r_2O'  - (b - b') \Kdelta \mathperiod
%\end{align*}
%\noindent where $\pi' = (O', U', V')$.

The next algorithms $ \SpecialG.\VerifyCom $ and $ \SpecialG.\Verify $ are identical to $ \ccgroth $ commitment and proof verification algorithms, respectively.
\item $\SpecialG.\VerifyCom(\pp, X, \barx, b) \rightarrow 0/1$: It outputs 1 iff
$X = \sum_{i=l+1}^{n} x_i K_i   + b \Kgamma$. 
\item $\SpecialG.\Verify(\crs, \bar{y}, X, \pi) \rightarrow 0/1$: It outputs  1 iff the following holds 
$e(A,B) = e(\alpha\gone, \beta\gtwo) + e(X + Y, \gamma \gtwo) + e(C, \delta \gtwo)$
where $\pi = (A, B, C)$, $Y = \sum_{i=1}^{l} x_i \delta_i\gone$ 
and $\bary = (x_1, \ldots, x_l)$. 

We remark that $ \SpecialG.\Verify $ corresponds to the verify algorithm of Groth16 zkSNARK when $ X $ is the output of $ \SpecialG.\Preprove$. 

\item$\SpecialG.\Sim(\tw, \bary, \relRQ) \rightarrow (\pi, X)$: It samples $\tilde{x}, a, b \xleftarrow {\$} \F_p$ and let $\pi = (a\gone, b \gtwo , \allowbreak c\gone)$ where $c = \frac{ab - \alpha \beta - \sum_{i=1}^{l} x_i (\beta a_i(\tau)+ \alpha b_i(\tau)+ c_i(\tau))- \tilde{x}}{\delta}  $ and, 
by definition $\bary = (x_1, \ldots, x_l)$. Note that $\pi$ is a simulated proof for transparent input $\bary$ 
and commitment $X = \tilde{x} \gone$.
\end{itemize} 



%\noindent \paragraph{Notes:} First, the trusted setup required by \SpecialG is 
%an extension of that required by original Groth16~\cite{Groth16} by two additional 
%group elements $\Kgamma = [\frac{\eta}{\gamma}]_1$ and $\Kdelta = [\frac{\eta}{\delta}]_1$. 
%An identical trusted setup to that used by \SpecialG was also used in LegoSNARK~\cite[Fig.~22]{LegoSNARK} which defines 
%a commit-carrying SNARK based on Groth16. Second, our $\SpecialG.\Reprove$ algorithm uses a Groth16 re-randomisation 
%technique for the proof (see~\cite[Fig.~1]{RandomizationGroth16} or LegoSNARK~\cite[Fig.~22]{LegoSNARK}), 
%but, in addition, $\SpecialG.\Reprove$ also re-randomises $X$ which is a commitment to a slice of the public input; moreover, in terms of security 
%properties, we appropriately define the zero-knowledge for zk continuations such that even after iteratively applying 
%$\SpecialG.\Reprove$ zero-knowledge property is preserved for both the witness as well as the public input committed to in $X$.  \\


%{\color{red}Note that, if a trusted setup is used (for example the one described in~\cite{subversion_zk}) such that there 
%exist a public and efficient procedure for verifying it, then, by extending it with $\Kgamma$ and $\Kdelta$ (which is our 
%extension of the standard Groth16 trusted setup), the resulting setup remains publicly verifiable (i.e., by using the additional 
%verification equation $e(\Kgamma, [\gamma]_2) = e(\Kdelta, [\delta]_2)$), and, hence, according to~\cite{subversion_zk}, 
%subversion zero-knowledge. }
%From page 3 of ~\cite{subversion_zk}"We change Groth?s zk-SNARK by adding extra elements to the CRS so that the CRS will become 
%publicly verifiable; this minimal step (clearly, some public verifiability of the CRS is needed in the case the CRS generator 
%cannot be trusted) will be sufficient to obtain Sub-ZK. However, choosing which elements to add to the CRS is not straightforward 
%since the zk-SNARK must remain knowledge-sound even given enlarged CRS; adding too many (or just ?wrong?) elements to the 
%CRS can break the knowledge-soundness."

We  prove that $\SpecialG$ is a zero-knowledge continuation in the full version of the paper, which can be found at this anonymous link~\cite{anonymous}.
%\noindent Next, we  prove that $\SpecialG$ is a zero-knowledge continuation. 
\eprint{We show that the knowledge soundness property 
of $\SpecialG$ (i.e., as defined for $\ZKCont_{\relone}$) is implied by the knowledge soundness property of commit-carrying SNARK with double 
binding (cc-SNARK with double binding, see Definition 3.4~\cite{LegoSNARK}); our notion of zero-knowledge for $\ZKCont_{\relone}$ is, in fact a 
new and stronger notion so we prove that directly. Formally, we have: 
}{}
\begin{theorem}
\label{sec_specialg}
Let $\relRQ$ be a relation as related to a QAP such that additionally $\{a_k(X)\}_{k=0}^n$ are linearly independent polynomials. Then, in the 
algebraic group model (AGM)~\cite{Fuchs_AGM}, $\SpecialG$ is a zero-knowledge continuation as per Definition~\ref{def:zk_cont}. 
\end{theorem}

\begin{comment}
\begin{proof} It is straightforward to show that $\SpecialG$ has perfect completeness for $\Preprove$ thanks to the completeness of the Groth16 zkSNARK because $ \SpecialG.\Verify $ is the same as the verification algorithm of the Groth16 zkSNARK when $ X $ is the output of $ \Preprove $.
For the perfect completeness for $\Reprove$, we have $ X', \pi' = (A',B',C') $ and $ b' $ given by the adversary and $ X,\pi =(A,B,C),b $ generated by $ \Reprove(crs, X',\pi',b') $. Clearly, if $ \SpecialG.\VerifyCom(pp,X',\barx,b') $ verifies, $ \SpecialG.\VerifyCom(pp,X,\barx,b) $ verifies. The verification of  $ X, \pi $ given that $ X',\pi' $ is verified becomes straightforward when we replace $ X $ with $ X' + (b-b')\Kgamma $, $ A $ with $ \frac{A'}{r_1} $, $ B $ with $ r_1 B' + r_1r_2 \delta \gtwo$ $C$ with $ = C' + r_2A'  - (b - b') \Kdelta $. After the replacement, the equality check reduced to $ e (A',B') = e(\alpha\gone,\beta\gtwo) + e(X'+Y, \gamma \gtwo) + e(C', \delta\gtwo)$ which is the verification check for $ X', \pi' $. So, if the adversarial proof verifies then the proof generated by $ \Reprove $ verifies as desired from the perfect completeness property of $ \Reprove $.

We next prove knowledge-soundness (KS) as in Definition~\ref{def:zk_cont} by first arguing $\SpecialG$ is a cc-SNARK with double binding 
(see Definition 3.4~\cite{LegoSNARK}).  We use the fact that $\ccgroth$ as defined by the NILP detailed in Fig.22, Appendix H.5~\cite{LegoSNARK} 
satisfies that latter definition. Moreover, $\SpecialG$'s $\Setup$ on one hand, and $\ccgroth$'s $ \mathsf{KeyGen} $, on the other hand, are the same 
procedure. Also $\SpecialG$ and $\ccgroth$ share the same verification algorithm. Hence, translating the notation appropriately, $\SpecialG$ also 
satisfies KS of a cc-SNARK with double binding. 

% we first argue that We prove knowledge-soundness by reducing it to the knowledge-soundness property of the Groth16 
%commit-carrying with double binding scheme (for short $\ccgroth$). This knowledge-soundness property has been formalised in 
%Definition 3.4~\cite{LegoSNARK} and the NILP corresponding to $\ccgroth$ has been detailed in Fig.22, Appendix H.5~\cite{LegoSNARK}. 
%Indeed, $\SpecialG$'s $\Setup$ together with $\Gen$ and $\ccgroth$'s $\mathit{KeyGen}$ are the same procedure. Moreover $\SpecialG$ 
%and $\ccgroth$ share the same verification algorithm. Since $\ccgroth$ satisfies the definition of a cc-SNARK with double binding, translating 
%the notation appropriately, $\SpecialG$ also satisfies the knowledge soundness properties for a cc-SNARK with double binding. 
Let $\adv_{\SpecialG}$ be an adversary for KS in Definition~\ref{def:zk_cont} and 
define adversary $\adv_{\ccgroth}$ for KS in Definition 3.4~\cite{LegoSNARK}:
\eprint{\begin{align*}
&\mathit{If} \ \adv_{\SpecialG} (\crs, \pp, \realaux, \relRQ)\ \mathit{outputs} \  (\bary, \barx, X, \pi, b) \\
&\ \ \ \ \ \ \ \ \mathit{then}\ \adv_{\ccgroth} (\crs, \realaux, \relRQ)\ \mathit{outputs} \ (\bary, X, \pi). 
\end{align*}}{If $ \ \adv_{\SpecialG} (\crs, \pp, \realaux, \relRQ) $ outputs   $ (\bary, \barx, X, \pi, b) $,then  $ \adv_{\ccgroth} (\crs, \realaux, \relRQ) $ outputs $ (\bary, X, \pi) $.}
Given extractor $\Extract_{\ccgroth}$ fulfilling Definition 3.4~\cite{LegoSNARK} for $\adv_{\ccgroth}$, we
construct extractor $\Extract_{\SpecialG}$ for $\adv_{\SpecialG}$.
\eprint{\begin{align*}
&\mathit{If} \Extract_{\ccgroth} (\crs, \realaux, \relRQ)\ \mathit{outputs} \ (\barx^*, b^*, \baromega^*) \\
& \ \ \ \ \ \ \ \ \mathit{then} \ \Extract_{\SpecialG} (\crs, \realaux, \relRQ) \ \mathit{outputs} \ \baromega^*; \\
& \mathit{Otherwise} \ \Extract_{\ccgroth}(\crs, \realaux, \relRQ) \ \mathit{outputs} \ \bot.
\end{align*}}{If $ \Extract_{\ccgroth} (\crs, \realaux, \relRQ) $ outputs  $ (\barx^*, b^*, \baromega^*) $ 
then $ \Extract_{\SpecialG} (\crs, \realaux, \relRQ) $ outputs  $ \baromega^* $; otherwise $ \Extract_{\ccgroth}(\crs, \realaux, \relRQ) $ outputs $ \bot $.}

\noindent We show $\Extract_{\SpecialG}$ fulfils Definition~\ref{def:zk_cont} for $\adv_{\SpecialG}$. Assume by contradiction that this does not hold.
This implies there exists an auxiliary input $\realaux$ such that each: 
\eprint{\begin{align*}
&\SpecialG.\Verify(\crs, \bary, X, \pi, \relRQ) =1 \ \ (10), \ \SpecialG.\VerifyCom(\pp, X, \barx, b) =1 \ \ (20), \ (\bary, \barx; \baromega) \notin \relRQ  \ \ (30) 
\end{align*}}{
$ \SpecialG.\Verify(\crs, \bary, X, \pi, \relRQ) =1 \ \ (10), \SpecialG.\VerifyCom(\pp, X, \barx, b) =1 \ \ (20),  (\bary, \barx; \baromega) \notin \relRQ  \ \ (30)  $
}
\liz{Where are equations (10), (20), etc.?}
holds with non-negligible probability. Since $(20)$ holds with non-negligible probability and verification (for both proofs and commitments actually) is identical in $\SpecialG$ and $\ccgroth$ respectively, 
and since $\Extract_{\ccgroth}$ is an extractor for $\adv_{\ccgroth}$ as per Definition 3.4~\cite{LegoSNARK},
 then each of the two events 
$\mathit{\ccgroth.\mathit{VerCommit}^*}(\mathit{ck}, X, \barx^*, b^*) =1 \ (40), \ (\bary, \barx^*; \baromega^*) \in  \relRQ \ (50)
$
 holds with overwhelming probability. Since $(20)$ holds with non-negligible probability and $(40)$ holds with overwhelming probability and 
together with (ii) from Definition 3.4~\cite{LegoSNARK} we obtain that $\barx^* = \barx$. Since $(50)$ holds with overwhelming probability, it implies 
$(\bary, \barx; \baromega^*) \in \relRQ $ with overwhelming probability, which contradicts our assumption.  Thus, our claim that $\SpecialG$ does not have 
KS as per Definition~\ref{def:zk_cont} is false. 

Finally, regarding zero-knowledge, it is clear that if $\pi = (A, B, C)$ is part of the output of $\SpecialG.\Reprove$, 
then $A$ and $B$ are uniformly distributed as group elements in their respective groups. This holds, as long as the input to $\SpecialG.\Reprove$ is a verifying proof, even when the proof was maliciously generated. Hence, it is easy to check  
that the output $\pi'$ of $\SpecialG.\Sim$ is identically distributed to a proof $\pi$ output by $\SpecialG.\Reprove$ so the perfect 
zero-knowledge property holds for $\SpecialG$. 
\end{proof}
\end{comment}

% !TEX root = main.tex

\subsection{Putting Together a NIZK and a $\ZKCont$  for Proving $\rel$} \label{sec:nizkR}

%TODO Check consistency with the preliminaries section
%$$\rel = \{(\bary, \barz; \barx, \baromega_1, \baromega_2):  (\bary, \barx; \baromega_1) \in \relone, (\barz, \barx; \baromega_2) \in \reltwo \},$$
Let $\ZKCont_{\relone}$ be a zk continuation for $\relone$ (from the preamble of this section) \liz{Where?} with public parameters $ \pp $ and 
let $\nizktwo$ be a NIZK for $\reltwo'$ defined as
$$\reltwo' = \{(X, \barz; \barx, b, \baromega_2): \ZKCont_{\relone}.\VerifyCom(\pp, X, \barx, b) =1\ 
\wedge \ (\barz, \barx; \baromega_2) \in \reltwo \},$$ 
with $\reltwo$ from the preamble of Section~\ref{sec:rvrf_cont}. Then we define the system $\nizkR$ for relation $\mathcal{R}$ 
from the preamble of this section as:
\begin{itemize}
\item $\nizkR.\Setup(1^{\lambda})\rightarrow (\crsR = (\crs,\crstwo), \twR = (\tw, \twtwo), \pp_\rel = \pp)$: Here,
$(\crs, \tw, \allowbreak \pp) \leftarrow \ZKCont_{\relone}.\Setup(1^{\lambda}, \relone)$, $(\crstwo, \twtwo) \leftarrow \nizktwo.\Setup(1^{\lambda})$.

%\item $\nizkR.\Gen: (\crsR) \mapsto (\ppR = \pp, \crspkR = (\crspk,\crspktwo), \crsvkR = (\crsvk,\crsvktwo))$ where 
%$(\crspk, \crsvk, \pp) \leftarrow \ZKCont_{\relone}.\Gen(\crs, \relone)$, \\ $(\crspktwo, \crsvktwo) \leftarrow \nizktwo.\Gen(\crstwo)$ 

\item $\nizkR.\Prove(\crsR,\bary, \barz; \barx, \baromega_1, \baromega_2 ) \rightarrow (\pi_1, \pi_2, X)$: Here 
$(X', \pi'_1, b')$ is generated by $\ZKCont_{\relone}.\Preprove(\crs, \allowbreak \bar{y}, \allowbreak \bar{x}, \baromega_1)$ and 
$(X, \pi_1, b)$ generated by  $\ZKCont_{\relone}.\Reprove(\crs, X', \allowbreak \pi_1', b')$ and $ \pi_2 $ is generated by
$ \nizktwo.\Prove(\crstwo, \allowbreak X, \allowbreak \barz; \allowbreak \barx, b, \baromega_2)$. 

\item $\nizkR.\Verify(\crsR,(\bary, \barz), (\pi_1, \pi_2, X)) \rightarrow 0/1$: It outputs $1$ iff 
$\ZKCont_{\relone}.\Verify(\crs, \allowbreak \bar{y}, \allowbreak X, \pi_1) = 1 $ and $ \nizktwo.\Verify(\crstwo, X, \barz, \pi_2) =1.$

\item $\nizkR.\Sim: (\twR, \bary, \barz) \mapsto (\pi_1, \pi_2, X)$ where 
$(\pi_1, X) \leftarrow \ZKCont_{\relone}.\Sim(\tw, \bary)$, $\pi_2 \leftarrow \nizktwo.\Sim(\twtwo, X, \barz)$.
 \end{itemize}
%\vspace{-0.5cm}

 
\begin{theorem}
	If $\ZKCont_{\relone}$ is a zk continuation for $\relone$ and $\nizktwo$ is a NIZK for $\reltwo'$ for some appropriately chosen public parameters $\pp$, 
	then the $\nizkR$ construction described above is a NIZK for $\rel$.
\end{theorem}

\textit{Proof Outline.} The correctness, knowledge soundness and zk properties of $ \nizkR $ come from the same properties of $ \ZKCont_{\relone} $ and $ \nizktwo $. 
The proof can be found in the full version of the paper, at this anonymous link~\cite{anonymous}.
%The proof can be found in Appendix \ref{ap:nizkR}.

% !TEX root = main.tex

\section{Our Second Ring VRF Construction Based on $ \ZKCont $}

\label{subsec:rvrf_faster}
We enhance our construction from Section \ref{sec:pedersen_vrf} by incorporating $ \ZKCont $. This protocol leverages the rerandomization properties of $ \ZKCont $, allowing a signer to generate a signature for a different input $ \msg $ within the same ring without having to recompute the most expensive part of the NIZK proof related to the key membership of the ring. $\CommitRing, \OpenRing $ are the same as in the first ring VRF protocol and $ \KeyGen$ is as in the alternative-1 of our first protocol.
\begin{itemize}
	\item $ \rVRF.\Setup(1^\secparam)   $ outputs $ \pprvrf = (crs_{\relinner}, \crsnark,  \pp_{\relinner}, (p, \grE,\genG, \genB), \F_p) $ where $ \crs_{\relinner}$ and $\pp_{\relinner} $ are generated by $ \ZKCont.\Setup(1^\secparam, \relinner) $ and $ \crsnark $ is generated by $ \NARK_{\relcomring}.\Setup(1^\secparam) $.
	
	\item $\rVRF.\rSign(\sk,\comring, \openring,\msg,\aux) $ computes $ \PreOut = \sk H_\grE(\msg) $, lets $ \Out = H(\msg, \PreOut) $. Then runs $ \NIZK_{\relrvrf}.\Prove(\comring, \allowbreak \PreOut, \allowbreak \msg, \allowbreak \Out; \sk,\openring)\rightarrow \pi_{rvrf}$ where
	
\eprint{$$ \relrvrf = \Setst{
		\begin{aligned}
			&\comring, (\PreOut,\msg, \Out); \\
			&x,r, \openring
		\end{aligned}
	}{
		\begin{aligned}
			& (\comring,x;r,\openring) \in \relinner, \\
			& (\PreOut, \msg, \Out; x) \in \relout
	\end{aligned}} \text{ and}$$}
{\begin{footnotesize}
		$$ \relrvrf = \Setst{
			\begin{aligned}
				&\comring, (\PreOut,\msg, \Out); \\
				&x,r, \openring
			\end{aligned}
		}{
			\begin{aligned}
				& (\comring,x;r,\openring) \in \relinner, \\
				& (\PreOut, \msg, \Out; x) \in \relout
		\end{aligned}} \text{ and}$$
	\end{footnotesize}
}   
\eprint{
	and
	$$ \relinner = \Setst{ (\comring, x ; r,\openring) }{
		\begin{aligned}
			&	\pk = \rVRF.\OpenRing(\comring,\openring), \\
			& 	x = \mathsf{Com}.\mathsf{Open}(\pk;x, r) 
		\end{aligned}	
	},$$
	$$\relout = \{((\PreOut, \msg,\Out),\sk; \perp): \PreOut = \sk H_\grE(\msg), \Out = H(\msg,\PreOut)\}$$
}{
\begin{footnotesize}
	$$ \relinner = \Setst{ (\comring, x ; r,\openring) }{
		\begin{aligned}
			&	\pk = \rVRF.\OpenRing(\comring,\openring), \\
			& 	x = \mathsf{Com}.\mathsf{Open}(\pk;x, r) 
		\end{aligned}	
	},$$
	$$\relout = \{((\PreOut, \msg,\Out),\sk; \perp): \PreOut = \sk H_\grE(\msg), \Out = H(\msg,\PreOut)\}$$
\end{footnotesize}
}
	
	We instantiate $ \NIZK_{\relrvrf}.\Prove $ as described in Section \ref{sec:nizkR} where $ \relone = \relinner $ and $ \reltwo' = \rel_{eval} $ and $\baseR = \relrvrf$ . \eprint{It works as follows: It runs $ \ZKCont_{\relinner}.\Preprove(crs_{\relinner}, \comring,x,r,\openring) $ and obtains $ \compk', \pi' , \openpk' $. Then, it runs $ \ZKCont_{\relinner}.\Reprove(crs_{\relinner}, X',\pi',\openpk') $ and obtains $ (\compk, \pi_{\mathtt{inner}}, \openpk) $. Finally, it lets $ \PreOut = x H_\grE(\msg) $ and runs $ \NIZK_{\rel_{eval}}.\Prove(\compk, \PreOut, \msg;x,\openpk) \rightarrow \pieval $ as described in Section $ \ref{sec:pederson_vrf} $ with $\aux' = \tmpaux$ .}{}
	In the end, it returns the ring VRF signature $ \sigma = (\compk, \pi_{\mathtt{inner}},\comring, \pieval, \PreOut)$\eprint{}{ where $ \compk,\pi_{\mathtt{inner}}, \openpk $ generated by $\ZKCont_{\relinner}.\Prove $ and $ \pieval $ is generated by $ \NIZK_{\rel_{eval}}.\Prove $. Here, $ \compk $ is a Pedersen commitment to $ x $ as described in $ \PedVRF $} in Section~\ref{sec:pedersen_vrf}.
	
	\item  $\rVRF.\rVerify : (\comring,\msg,\aux,\sigma) \mapsto (1,\Out) \,\, \lor (0,\perp)$ \,
	it parses $\sigma$ as $(\compk, \allowbreak \pi_{\mathtt{inner}}, \allowbreak \comring, \allowbreak \pieval, \PreOut)$ and runs  $\NIZK_{\relrvrf}.\Verify(\comring, \PreOut, \msg, \Out; \pieval,\compk, \pi_{\mathtt{inner}})$  \eprint{i.e., runs $ \ZKCont_{\relinner}.\Verify(crs_{\relinner}, \allowbreak \comring, \allowbreak \compk, \allowbreak \pi_{\mathtt{inner}}) $ and $ \NIZK_{\rel_{eval}}.\Verify(\compk, \allowbreak \PreOut,\allowbreak  \msg; \pieval) $}{as described in Section \ref{sec:nizkR} }. If all verify, it outputs $ (1,\Out = H(\msg,\PreOut)) $. Otherwise, it returns $ (0,\perp) $.
\end{itemize}




\begin{theorem}\label{thm:rvrfspecial}
	Our specialized $ \rVRF $   realizes $ \fgvrf $ running $ \Gen_{sign} $ (Algorithm \ref{alg:gensignSG}) \cite{canetti1,canetti2} in the random oracle model assuming that $\ZKCont $ is zero-knowledge and knowledge sound as defined in Definition \ref{def:zk_cont}, $ \NIZK_{\mathcal{R}_{eval}} $ is zero-knowledge and knowledge sound, $ \NARK_{\relcomring} $ is knowledge sound, the DDH problem is hard in $ \grE  $, %\eprint{(so the CDH problem is hard as well)}{},
	and the commitment scheme $ \mathsf{Com} $ is binding and perfectly hiding. 
\end{theorem}
\begin{algorithm}
	\eprint{}{\scriptsize}
	\caption{$\Gen_{sign}(\ring, \sk= (x,r),\pk,\aux,\msg)$}
	\label{alg:gensignSG}	 	
	\begin{algorithmic}[1]
		\State $ \comring, \openring \leftarrow \rVRF.\CommitRing(\ring, \pk) $
		\State $ \compk', \pi', \openpk' \leftarrow \ZKCont.\Preprove(crs, \comring,\sk,(r,\openring)) $
		\State $ \compk, \pi_{\mathtt{inner}}, \openpk \leftarrow \ZKCont\Reprove(crs, \compk',\pi',\openpk') $ 
		\State $ c,s_1, s_2 \leftsample \F_p $
		\State $ \pi_{eval}  \leftarrow (c,s_1, s_2)$		
		\State\Return$ \sigma = (\compk, \pi_{\mathtt{inner}},\comring, \pieval) $
	\end{algorithmic}
	
\end{algorithm}

\noindent \textit{Proof Outline.}  The security proof is very similar to that of our first construction.
We construct the same $ \simulator $ described in the proof of Theorem \ref{thm:firstprotocol} because in the second construction random oracles are the same as in the first construction. We then use the result of Lemma \ref{lem:anonymity} because $ \PreOut $ is the same. The only difference is in Lemma \ref{lem:simulation-ind}, since $ \Gen_{sign} $ is different than Algorithm \ref{alg:gensign}. There, we run the simulator and extractor of $ \NIZK_{\relrvrf} $  instead of extractors of  $ \NIZK_{\Rring} $ and $ \NIZK_{\rel_{eval}} $. The proof can be found in Appendix \ref{ap:secondprotocolproof}.



%requires two scalar multiplications on $\grE_1$
%and two on the same or faster $\grE'$,
%so together with $\SpecialG.\Reprove$ costing four scalar multiplications
%on $\grE_1$ and two on $\grE_2$, our amortised prover time
%runs faster than 12 scalar multiplications on typical $\grE_1$ curves. 
%We expect the three pairings dominate verifier time, but
%verifiers also need five scalar multiplications on $\grE_1$.

%TODO WHAT DOES IT MEAN
%Importantly, our fast ring VRF's amortised prover time now rivals
%group signature schemes' performance \cite{group_sig_survey}.
%We hope this ends the temptation to deploy group signature like
% constructions where the deanonymisation vectors matter. 

