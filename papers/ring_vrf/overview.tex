% !TEX root = crypto.tex

\subsection{Technical Overview}
\label{sec:overview}

\liz{This section needs to be substantially shortened. Move details elsewhere.}

To begin with, we introduce a ring VRF interface and propose a ring VRF protocol realizing the desired security properties.
We then provide intuition for our second ring VRF construction, which features some amortization techniques.
Similar to a VRF \cite{vrf_micali}, a ring VRF includes:

\begin{itemize}
\item $\rVRF.\KeyGen: $ outputs a secret key and public key $ (\sk, \pk)$.

\item $\rVRF.\Eval(\sk,\msg) \mapsto \Out$:  \emph{deterministically} computes the VRF output \Out from a secret key \sk and an input \msg.
\end{itemize}
%
% Although many constructions exist,
%Our \rVRF.\KeyGen and \rVRF.\Eval initially resemble EC VRFs like \cite{nsec5,VXEd25519,draft-irtf-cfrg-vrf-10}.
% In other words,
% internally we prove a VUF output $\PreOut = \sk H_{\grE}(\msg)$,
% with a hash-to-curve $H_{\grE}$, so then applying a PRF $\Hout$ yields a
% VRF output $\Out = \Hout(\msg, h \PreOut)$ ala \cite[Prop. 1]{vrf_micali},
% using a key pair like $\pk = \sk \genG$ for a generator $\genG$.

%We demand pseudo-randomness properties from \Eval, which could mirror
%\cite{vrf_micali} if desired.  We provide a UC definition resembling
%\cite{praos,ucvrf} which handles adversarial keys better however. %NOT CLEAR WHAT HANDLING BETTER MEANS

Ring VRFs have the following important pseudorandomness property: the output of \Eval must be pseudorandom for all $ \sk $, including those that are maliciously generated.
%\eprint{In our construction in \S\ref{sec:pederson_vrf},  \rVRF.\KeyGen and \rVRF.\Eval resemble EC VRF like \cite{nsec5,VXEd25519,draft-irtf-cfrg-vrf-10}.}{}

% TODO: Should this text be moved elsewhere?
% and prove it corresponds to $\rVRF.\Eval$ for some plausible signer.

% As an instructive but insecure over simplification, 
In contrast to VRFs, a ring VRF scheme has the following algorithms operating directly upon
 set of public keys \ring:
\begin{itemize}
\item $\rVRF.\rSign(\sk,\ring,\msg, \aux) \mapsto \sigma:$
    returns a ring VRF signature $\sigma$ which is a proof for the ring VRF output of $ \msg $ as well as a signature signing $ \aux $.
\item $\rVRF.\rVerify(\ring,\msg,\aux,\sigma) \mapsto \Out \, \lor \perp:$
    returns either an output $\Out$ or else failure $\perp$. It returns $ \Out $ if $ \sigma  $ signs $ \aux $ with one of the keys in $ \ring $ and $ \Out $ is the ring VRF output of $ \msg $ and the same key that signs $ \aux $.
\end{itemize}


Ring VRF protocols deviate from VRF protocols in that the public key of the signer is not known during the verification process. Instead, a set of public keys is given, including the signer's key, as in ring signatures. However, ring VRFs distinguish themselves from ring signatures  in the verification process as well in which the unique ring VRF output of the signer is revealed if the signature is successfully verified with $\ring$. \liz{?}
In essence, a verified ring VRF signature of an input  $ \msg $ actually proves that $ \Out $ is the evaluation output of $ \msg $ generated by the signer's key. 

Anonymity is a key property of ring VRFs, ensuring that the verifier learns nothing about the signer except that the  evaluation value of the signed input $ \msg $ is $ \Out $ and that the signer's public key is in $ \ring $.
Consider the following intuitive ring VRF scheme.
Let \rVRF.\Eval be a pseudorandom function, and use a NIZK protocol $ \NIZK $, where $ \rVRF.\Sign $ runs the proving algorithm and $ \rVRF.\Verify $ runs the verification algorithm of $ \NIZK $ for a relation consisting of statement and witness pairs, as follows:
%\vspace{-2mm}
\doublecolumn{
	\begin{scriptsize}
		$$ \rel_{\mathsf{rvrf}} = \Setst{ (\Out, \msg, \ring);(\sk,\pk)}{
		\begin{aligned}
			& (\pk,\sk) \leftarrow \rVRF.\KeyGen,\\
			& \pk \in \ring \\
			& \Out = \rVRF.\Eval(\sk,\msg) 
		\end{aligned}
		} $$
	\end{scriptsize}
}{
	$$ \rel_{\mathsf{rvrf}} = \Setst{ (\Out, \msg, \ring);(\sk,\pk)}{
	\begin{aligned}
		& (\sk,\pk) \leftarrow \rVRF.\KeyGen,\\
		& \pk \in \ring \\
		& \Out = \rVRF.\Eval(\sk,\msg)
	\end{aligned}
} $$
}



% TODO:  \PRF vs \rVRF.\Eval here??
% Although convenient for security arguments, % formalization


The zero-knowledge property of the NIZK ensures that the verifier learns nothing about the specific
signer, except that their public key is in the ring and maps $\msg$ to $\Out$.
Importantly, pseudorandomness also guarantees that \Out is an anonymous identity
for the specific signer, but only within the context of \msg. \liz{We may want to use this language of context elsewhere.} We ignore $ \aux $ in $ \rel_{\mathsf{rvrf}} $ for the sake of simplicity. Otherwise, we note that it is imperative to incorporate \rVRF.\Sign and \rVRF.\Verify to sign associated data \aux. \liz{?}


%In addition to the primary task of proving an evaluation value anonymously, it is imperative to incorporate \rVRF.\Sign and \rVRF.\Verify to sign associated data \aux. Neglecting this  would leave ring VRF signatures vulnerable, rendering them susceptible to replay attacks, particularly in identity applications. \eprint{As an example, our identity protocol below in \S\eprint{\ref{sec:app_identity}}
%	yields the same ring VRF outputs each time the same user logs into the
%	same site, which suffers replay attacks unless \aux binds the
%	ring VRF signature to the TLS session.}{}
% \smallskip


% \smallskip

If one were to use the ring VRF interface described above, then one would need time
$O(|\ring|)$ in \rVRF.\rSign and \rVRF.\rVerify merely to read their \ring
argument, which severely limits applications.
Instead, we replace $ \ring $ with a commitment to the ring, such as Merkle tree root, and run asymptotically faster. Therefore, we introduce the following algorithms for $ \rVRF $:
\begin{itemize}
% \item $\rVRF.\CheckRing : \ring \mapsto \comring$ takes a set \ring of public keys and returns a public key set commitment \comring.
\item $\rVRF.\CommitRing : (\ring,\pk) \mapsto (\comring,\openring)$ \,
    returns a commitment for a set \ring of public keys, and
     the opening \openring if $\pk \in \ring$.
\item $\rVRF.\OpenRing : (\comring,\openring) \mapsto \pk \, \lor \perp$ \,
    returns a public key \pk, provided \openring correctly opens
    the ring commitment \comring, or failure $\perp$ otherwise.
\end{itemize}

Together with these algorithms, we can replace the membership condition $\pk \in \ring$ in $  \rel_{\mathsf{rvrf}} $ by the opening condition
$ \pk = \rVRF.\OpenRing(\comring,\openring) $ and replace $ \ring $ in the statement with $ \comring $ and add $ \openring $ to the witness.
\liz{Committing to the ring using a Merkle tree is not new, so I don't think we need to go into this depth. We want to highlight the efficiency of reproving instead.}
%
%\eprint{Indeed, regular (non-anonymous) VRF uses always encounter similar tension
%	with VRF inputs \msg being smaller than full message bodies $(\msg,\aux)$.
%	As an example, Praos \cite{praos} binds their VRF public key together
%	with a second public key for another (forward secure) signature scheme,
%	with which they sign their \aux, the block itself.
%	%
%	An EC VRF should expose an \aux parameter which it hashes when computing
%	its challenge hashes.  Aside from saving redundant signatures, exposing
%	\aux avoids user key handling mistakes that create replay attacks.}{}
%
%\eprint{Ring VRFs cannot so easily be combined with other signatures, which
%	makes \aux essential,%
%	\eprint{\footnote{If ring VRFs authorized creating blocks in an anonymous Praos blockchain then \aux must include the block being created, or else others could steal their block production turn.}}{}
%	but thankfully our ring VRF construction in \S\ref{sec:pederson_vrf} exposes \aux exactly like EC VRFs should do.%
%	\eprint{\footnote{We suppress multiple input-output pairs until \S\ref{subsec:multi_io} below, but they work like in \cite{PrivacyPass} too.}}{}}{}
%
% $\pk = \OpenRing(\comring,\openring)$.
%
% $$ \pi_0 = \NIZK \Setst{ \Out, \msg, \comring }{
%     \begin{aligned}
%         \exists (\pk,\sk) &\leftarrow \KeyGen,  \quad
%           \Out = \PRF(\sk,\msg)  \\
%         \exists \openring \textrm{\ s.t.\ }
%           \pk &= \OpenRing(\comring,\openring)  \\      
%     \end{aligned}
% } $$
%
% \smallskip
%
%\eprint{Addressing these concerns, our notion should really be named 
% \emph{ring verifiable random function with additional data}
%and its basic methods look like
%\begin{itemize}
%\item $\rVRF.\rSign : (\sk,\openring,\msg,\aux) \mapsto \sigma$, \quad and
%\item $\rVRF.\rVerify : (\comring,\msg,\aux,\sigma) \mapsto \Out \,\, \lor \perp$.
%\end{itemize}}{}
\\\\
\noindent \textbf{Our Approach.} Although an asymptotic improvement with $ \rVRF.\CommitRing $, 
the intuitive scheme can be computationally expensive to prove the evaluation value in combination with the ring membership condition. Therefore, in our first ring VRF construction,  we divide the relation into two relations. The first relation $ \Reval $ is designed  to show the validity of the evaluation value with a proof that can be efficiently generated using discrete logarithm equality proofs. \liz{This is very out of context. We need to talk about how we use ECVRF.} We integrate $ \aux $ in this proof so that the proof serves as a signature on $ \aux $ signed by the same key used to generate the ring VRF output.  \liz{It is not clear why we are signing $\aux$.} The second relation $ \Rring $ is designed to show that the key used in the evaluation and in signing $ \aux $ is in the ring. Its statement therefore has one part of the evaluation proof which is the Pedersen commitment to the secret key in order to relate the key in $ \Rring $ and $ \Reval $.
\liz{This comes out of nowhere.}

The most computationally expensive part of our first protocol is generating a proof for $ \Rring $ during signing. Therefore, we deploy a further optimization for our second ring VRF construction. For this,  we consider optimizing the cases where a party generates ring VRF signatures of different inputs for the \emph{same ring}. \liz{Mention scenarios again.} In this case, the main change in the new signature is caused by the cheapest  \liz{Alternative word?} part of the signing process, which generates a proof for the ring VRF output and signature for $ \aux $  because the input changes so the evaluation value changes. \liz{Reword.} Consequently, it raises the question of why a party should re-execute the ring membership proving.  In light of this, we deploy in the second protocol our  new notion of a \emph{zk-continuation} that allows us to generate proofs for $ \rel_{\mathsf{rvrf}} $ by reusing a previously generated proof for ring membership with simple operations. 
In a nutshell, if a party once generates a ring VRF signature for $ \ring $ in our second scheme, this signature has a  Pedersen commitment to the secret key $ \sk $ i.e., $ \compk = \sk \genG + \openpk \genB $ as a part of the proof for the output, similar to our first scheme. \liz{First scheme not described.} When this party generates another signature for a different input  with $ \ring $, they (re)randomize the  existing proof for the ring membership with a random number $ \openpk' $ rather than  running the proving algorithm for the ring membership from scratch. Then, they generate a new proof for the ring VRF output by running the proving algorithm for the ring VRF output by setting the new Pedersen commitment to the secret key with $ \compk' = \sk \genG + \openpk'\genB $ which is the new  Pedersen commitment to the secret key generated  with the same randomness $ b' $ used in ring membership.



%demands some hiding commitment \compk to \pk.

%


%\def\tmpAA{\Out = \rVRF.\Eval(\sk,\msg)}%
%\def\tmpBB{\textrm{\compk commits to\ \sk}}%
%$$ \rel_{eval} = \Setst{ (\Out, \msg, \aux, \compk); \sk}{
%	\eprint{
%		\tmpAA, \, \tmpBB
%	}{
%		\begin{aligned}
%			&\tmpAA, \\
%			&\tmpBB \\
%		\end{aligned}
%	}
%} $$
%
%\def\tmpAA{\textrm{\compk commits to $ \sk $ with public key\ }}%
%\def\tmpBB{\rVRF.\OpenRing(\comring,\openring)}%
%$$ \Rring = \Setst{ (\compk, \comring);(\sk,\pk) }{
%	\eprint{
%		\tmpAA \pk = \tmpBB
%	}{
%		\begin{aligned}
%			&\tmpAA \\
%			&\, \pk = \tmpBB \\
%		\end{aligned}
%	}
%} $$

%TODO: WE SHOULD EXPLAIN IT BETTER FOR EPRINT
%We discovered the SNARK for the language \Lring becomes incredibly efficient for the prover if one specializes
%the original Groth16 SNARK construction:  An inner original Groth16 SNARK for $\Lring^\inner$
%handles the secret key \sk directly via its public inputs, but
%\sk and even \pk remain secret by transforming the trusted setup to have
%a rerandomizable Pedersen commitment \compk outside this Groth16 SNARK.
%$$ \Lring^\inner = \Setst{ \sk, \comring}{
%    \eprint{
%    (\pk,\sk) \leftarrow \rVRF.\KeyGen, \, % \textrm{\,and }
%    \pk = \rVRF.\OpenRing(\comring,\openring) 
%    }{
%    \begin{aligned}
%        &(\pk,\sk) \leftarrow \rVRF.\KeyGen, \\
%        % \exists \openring \textrm{\ s.t.\ }
%        &\pk = \rVRF.\OpenRing(\comring,\openring)  \\      
%    \end{aligned}
%    }
%} $$
%
%Our zero-knowledge continuation in \S\ref{sec:rvrf_cont} rerandomizes
%$\compk = \pk + b \, K$ without reproving the Groth16 SNARK for $\Lring^\inner$.
%For this, the secret key \sk must be a public input of $\Lring^\inner$, and
%the Groth16 trusted setup must be expanded by a secret multiple of
% the otherwise independent point $K$.
%
%In \S\ref{sec:pederson_vrf}, we introduce an extremely efficient NIZK
%for $ \rel_{eval} $, which also provides an essential proof-of-knowledge for \compk.


\endinput


% We define ring VRFs in \S\ref{sec:rvrf_games} and \S\ref{sec:rvrf_uc_fun} below, but
Ring VRFs are firstly ring signatures broadly interpreted, in that they
prove an involved public key lies inside some commitment \comring to
the plausible signer set, known as the ring.
Anyone could compute \comring from this set of public keys.
%
At the same time, ring VRFs prove correct output of a PRF keyed by
the signer's actual secret key, and evaluated on a supplied message \msg,
which then links ring VRF signatures on the same \msg.

\smallskip
